%!TEX root = ../my_thesis.tex
\chapter{Introduction}

In 1986, a group of scientist from IBM research developed the first Atomic Force Microscope.\cite{binnig1986atomic} The original idea is to measure forces between a sharp tip and a sample. Using a XY-scanner to move the sample on the horizontal plane allows to acquire data on different areas of the sample and generate images. 

The most conventional way to scan is with a raster pattern. Because of its limited bandwidth, fast raster scans generate distortions in the image.\cite{yong2010high} Spiral pattern allows to generate high-quality images at higher scan frequencies than the raster one. \cite{mahmood2009fast}. Moreover, this method reduces the number of sample acquired. \cite{nonrasterdata}

With raster scanning, the data is evenly distributed in space. Generating an image is trivial. Spiral scan, however, needs new techniques to render images. Fortunately, image processing algorithms like inpainting \cite{richard2001fast} or Delaunay triangulation have been developed to generate images from sparse data.\cite{nonrasterdata}.

The bandwidth on the z-axis control loop is limited by the dynamics of the z scanner.\cite{jeong:093706} We can achieve higher frequencies by using a small piezoelectric ceramic.\cite{sulchek1999dual}


In this thesis, we will see how we can use non-raster scan pattern to improve the bandwidth on the XY-plane. Also, we have developed image processing algorithms to render non-gridded data. Finally, we will investigate new ways to go beyond the limitations on the z-axis with tilt correction and dual actuators feedback on z.


