%!TEX root = ../my_thesis.tex
\chapter{Introduction}

In 1986, a group of scientist from IBM research developed the first Atomic Force Microscope.\cite{binnig1986atomic} The original idea was to sense forces between a sharp tip and a surface. The deflection of the cantilever is measured using a laser reflected from the top of the cantilever into a photodetector(array of photodiodes). Nowadays, AFM's applications range from measuring elastic properties of biological samples to imaging of surface topography\cite{radmacher1997measuring} \cite{hansma1988scanning}. The former technique is achieved by using a XY-scanner to move the sample on the horizontal plane. Therefore, the AFM acquires data on different points of the surface. Also, AFMs have a feedback system to maintain a constant tip-sample force. 

The most conventional way to scan is with a raster pattern. This technique steers the tip of the AFM on specific points of a grid. Past AFM research has been focused on improving position controllers. Indeed, the AFM precision depends on the quality of the closed loop feedback on XY. These improvements, however, didn't solve fundamental problems with raster scanning. Most of the data is thrown away (trace and retrace) and the actual position on the XY plane is inaccurate - and directly correlated with the efficiency of the position controller.

Because of its limited bandwidth, fast raster scans generate distortions in the image.\cite{yong2010high} Spiral pattern techniques generate high-quality images at higher scan frequencies than the raster one. \cite{mahmood2009fast}. Moreover, this method reduces the number of data points necessary. \cite{nonrasterdata}

With raster scanning, the data is evenly distributed in space so generating an image becomes trivial. Spiral scan, however, needs new techniques to render images. Fortunately, image processing algorithms like inpainting \cite{richard2001fast} or Delaunay triangulation have been developed to generate images from sparse data.\cite{nonrasterdata}.

The bandwidth on the z-axis control loop is limited by the dynamics of the z scanner.\cite{jeong:093706} We can achieve higher frequencies by using a small piezoelectric ceramic.\cite{sulchek1999dual}

In this thesis, we will see how we can use non-raster scan patterns to improve the bandwidth on the XY-plane. Also, we have developed image processing algorithms to render non-gridded data. We use the popular Open Graphics Library (OpenGL) for the rendering of the X, Y, Z data and investigate ways to use the Graphics Processing Unit (GPU) to improve the computing time.
Finally, we will implement new ways to go beyond the limitations on the z-axis with tilt correction and dual actuators feedback. If an image is tilted, you can use first-order plane fitting to flatten the image. A more efficient way is to dynamically compensate for the tilt of the sample. Also we will see how to improve the bandwidth on the z-axis by introducing a small piezoelectric ceramics.
