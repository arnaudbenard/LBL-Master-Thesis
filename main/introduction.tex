%!TEX root = ../my_thesis.tex
\chapter{Introduction}

Many methods exist to observe objects too small for the naked eye. Optical microscopy was one of the first developed technique (17th century) to see visible lights with a system of magnifying lenses. One of the advantage of optical microscopy is that you can record images at the video rate, but it still has a low resolution.
Electron microscope can acquire data faster because of the speed of the electron beam, but the raster scan pattern is a serious hindrance(speed). Atomic Force Microscope are slower than electron microscopes mainly because of its bulky mechanics (inertia). Toshi Ando /INSERT REF developed small fast piezos that can reach /INSERT VALUE. All of those techniques display data on a grid. We will see how we can use non-raster scan pattern to improve the bandwidth on the XY-plane. Also, we have developed image processing algorithms to render non-gridded data. Finally, we will investigate new ways to go beyond the limitations on the z-axis with tilt corrections and dual actuators feedback on z.



Limitations

Microscopy:
	- optical -> video rate but crappy resolution
	- EM -> raster scan can slow electron beam fast
	- AFM -> slow mostly because of bulky mechanics: intertia (mass of the system), toshi ando small piezos (small range) but fast

Using spiral has been shown to be better(demands less BW x,y) but need display of non gridded data (lower res. freq. system)

z is still limiting to go on large scales (x,y)


